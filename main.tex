\documentclass{article}
\usepackage{amsmath}
\usepackage{amssymb}

\title{Problem Set 1}
\author{Fernando, Dang, Raj, Eric M.}
\date{September 12, 2025}

\begin{document}

\maketitle

\section*{Problem 1}

\textbf{Statement:} Using a proof by contradiction, prove for any integer $n$ if $n^3 + 5$ is odd, then $n$ is even.

\textbf{Proof by Contradiction:}

Assume for contradiction that $n^3 + 5$ is odd and $n$ is odd.

Since $n$ is odd, we can write $n = 2k + 1$ for some integer $k$.

Then:
\begin{align}
n^3 &= (2k + 1)^3\\
&= 8k^3 + 12k^2 + 6k + 1\\
&= 2(4k^3 + 6k^2 + 3k) + 1
\end{align}

Let $m = 4k^3 + 6k^2 + 3k$. Since $k$ is an integer, $m$ is also an integer.

Therefore, $n^3 = 2m + 1$, which means $n^3$ is odd.

Now, $n^3 + 5 = (2m + 1) + 5 = 2m + 6 = 2(m + 3)$.

Since $m + 3$ is an integer, $n^3 + 5$ is even.

Any integer times 2 is even. Therefore, $n^3 + 5$ is even.

This contradicts the assumption that $n^3 + 5$ is odd.

Therefore, if $n^3 + 5$ is odd, then $n$ must be even.

\section*{Problem 2}

\textbf{Statement:} 

\[
\overline{A} \cap \overline{B} = \overline{A \cup B}
\]

\textbf{Example:} Let the universal set be 
\[
U = \{1,2,3,4,5\}
\] 

and let 
\[
A = \{1,2\}, \quad B = \{3,4\}.
\]

\textbf{Step 1: Find complements:}
\[
\overline{A} = U - A = \{3,4,5\}, \quad \overline{B} = U - B = \{1,2,5\}
\]

\textbf{Step 2: Solve left-hand side:}  
\[
\overline{A} \cap \overline{B} = \{3,4,5\} \cap \{1,2,5\} = \{5\}
\]

\textbf{Step 3: Solve right-hand side:}  
\[
A \cup B = \{1,2\} \cup \{3,4\} = \{1,2,3,4\}
\]  
\[
\overline{A \cup B} = U - (A \cup B) = \{5\}
\]

\textbf{Step 4: Compare both sides:}  
\[
\overline{A} \cap \overline{B} = \{5\} = \overline{A \cup B}
\]

\textbf{Conclusion:} The equality holds in this example, showing that it's true
\[
\overline{A} \cap \overline{B} = \overline{A \cup B}
\] 

\section*{Problem 3}

\textbf{Statement:} Using a proof by induction, show that for all $n > 5$, $2^n > 5n$.


\textbf{Proof by Induction:}

\textbf{Base Case:} Let $n = 6$.
$2^6 = 64 \quad \text{and} \quad 5 \cdot 6 = 30$
Since $64 > 30$, the base case holds.

\textbf{Inductive Hypothesis:} Assume that for some $k \geq 6$, $2^k > 5k$.

\textbf{Inductive Step:} We need to show that $2^{k+1} > 5(k+1)$.

$2^{k+1} = 2 \cdot 2^k > 2 \cdot 5k = 10k$
(by the inductive hypothesis)

We need to show that $10k > 5(k+1) = 5k + 5$.

This is equivalent to showing $10k > 5k + 5$, or $5k > 5$, or $k > 1$.

Since $k \geq 6 > 1$, we have $10k > 5k + 5$.

Therefore, $2^{k+1} > 10k > 5k + 5 = 5(k+1)$.

By mathematical induction, $2^n > 5n$ for all $n > 5$.

\section*{Problem 4}

\textbf{Statement:} Let $p$ and $q$ be truth values. Using a truth table, prove or disprove the following statement:
$$(\neg p \vee q) \wedge (p \wedge (p \wedge q)) \Leftrightarrow (p \wedge q)$$

\textbf{Truth Table:}

\begin{center}
    \begin{tabular}{|c|c|c|c|c|c|c|c|}
    \hline
    $p$ & $q$ & $\neg p$ & $\neg p \vee q$ & $p \wedge q$ & $p \wedge (p \wedge q)$ & LHS & LHS $\Leftrightarrow$ RHS \\
    \hline
    T & T & F & T & T & T & T & T \\
    T & F & F & F & F & F & F & T \\
    F & T & T & T & F & F & F & T \\
    F & F & T & T & F & F & F & T \\
    \hline
    \end{tabular}
    \end{center}
    
    Where LHS = $(\neg p \vee q) \wedge (p \wedge (p \wedge q))$ and RHS = $(p \wedge q)$.
    
    \textbf{Step-by-step calculation:}
    \begin{itemize}
    \item Row 1: $(\neg T \vee T) \wedge (T \wedge (T \wedge T)) = (F \vee T) \wedge (T \wedge T) = T \wedge T = T$
    \item Row 2: $(\neg T \vee F) \wedge (T \wedge (T \wedge F)) = (F \vee F) \wedge (T \wedge F) = F \wedge F = F$
    \item Row 3: $(\neg F \vee T) \wedge (F \wedge (F \wedge T)) = (T \vee T) \wedge (F \wedge F) = T \wedge F = F$
    \item Row 4: $(\neg F \vee F) \wedge (F \wedge (F \wedge F)) = (T \vee F) \wedge (F \wedge F) = T \wedge F = F$
    \end{itemize}
    

    
The biconditional is true in all rows, therefore the statement is \textbf{true}.

$(\neg p \vee q) \wedge (p \wedge (p \wedge q)) \Leftrightarrow (p \wedge q)$ is \textbf{true}.
    
\end{document}